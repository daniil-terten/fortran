\documentclass[a4paper,12pt]{article}
\usepackage{spbudiploma_tempora}
\usepackage[T2A]{fontenc}
\usepackage[utf8]{inputenc}
\usepackage[english,russian]{babel}
\usepackage{cmap}
\usepackage{geometry}

\usepackage{euscript}
\usepackage{longtable}
\usepackage{makecell}
\usepackage{listings}
\usepackage[pdftex]{graphicx}
\usepackage{amsthm,amssymb, amsmath}
\usepackage{textcomp}
\usepackage{spbudiploma_tempora}
\usepackage[T2A]{fontenc}
\usepackage[utf8]{inputenc}
\usepackage[english,russian]{babel}
\usepackage{cmap}
\usepackage{geometry}

\begin{document}
\begin{titlepage}
\title{3.1 Титульный лист}

\begin{center}
	Санкт-Петербургский политехнический университет Петра Великого\\
	Институт компьютерных наук и технологий\\
	\bfseries{Высшая школа программной инженерии}
\end{center}

\vspace{20ex} 

\begin{center}
	\begin{huge} {\MakeUppercase{\bfseries{\scshape курсовая работа}}} \end{huge}
	
	\vspace{3ex}
	
	{\bfseries Лабораторная работа 1}
	
	по дисциплине: «Алгоритмы и структура данных»
\end{center}

\vspace{30ex}

\noindent Выполнил\\
студент гр.в3530904/10022\hfill \begin{minipage}{0.6\textwidth} \hfill Полещук Е.О.\end{minipage}

\vspace{3ex}

\noindent Руководитель\\
старший преподаватель.\hfill \begin{minipage} {0.6\textwidth}\hfill Фёдоров С.А.\end{minipage}

\vspace{3ex}

\hfill \begin{minipage}{0.6\textwidth} \hfill «\underline{\hspace{0.1cm 23}}»\underline{\hspace{0.1cm мая}} 202\underline{\hspace{0.1cm 2}} г.\end{minipage}

\vfill

\begin{center}
	Санкт-Петербург\\ 
	2022
\end{center}
\end{titlepage}
\setcounter{page}{2}

\tableofcontents
\pagebreak

\begin{center}\specialsection{Введние}\end{center}
\textbf{Постановка задачи:}\\
Дан список группы в виде:
\begin{verbatim}
     ФАМИЛИЯ     ИМЯ     ОТЧЕСТВО
     15 симв.  10 симв.  15 симв.
\end{verbatim}
\begin{verbatim}
    Безруков Сергей Викторович
    Лебедев Пётр Станиславович
    Глухих Сергей Алексеевич
\end{verbatim}
Удалить из списка всех, у которых имена совпадают, кроме первого. Пример выходного файла:
\begin{verbatim}
     Безруков Сергей Викторович
     Лебедев Пётр Станиславович
\end{verbatim}
\textbf{Цель работы:}\\
Выбор структуры данных для решения поставленной задачи на современных микроархитектурах. 
Задачи:
\begin{enumerate}
	\item реализовать задание с использованием массивов строк;
	\item реализовать задание с использованием массивов символов и внутренних процедур головной программы;
	\item реализовать задание с использованием структуры массивов, файлов записей и модулей;
	\item реализовать задание с использованием стркутуры массивов, файлов записей, модулей и хвостовой рекурсии;
	\item реализовать задание с использованием модулей, хвостовой рекурсии, однонаправленных списков неизвестной длины;
	\item провести анализ на регулярный доступ к памяти;
	\item провести анализ на векторизацию кода;
	\item провести сравнительный анализ реализаций.
\end{enumerate}

Во всех вышеизложенных заданиях необходимо также использовать регулярное программирование.\\
\textbf{Указание:}\\
Работать с исходным массивом. При каждом нахождении повторяющегося имени будет неэффективным тут же его удалять и сдвигать все остальные элементы в массиве, потому что имя опять может встретиться.

Будет неэффективным нахождение сперва всех мест в массиве, где находится текущее имя, а потом удаление его оттуда сдвигом остальных элементов, потому что среди остальных имён могут оказаться другие повторяющиеся, которые всё равно в итоге придётся удалять.

Необходимо сперва сформировать массив-маску для всех уникальных имен. При этом не стоит проверять имя, если оно уже встречалось. Затем по маске упаковать исходные массивы (Pack).
\newpage

\begin{center}\specialsection{Основная часть}\end{center}
\section{Реализация и анализ применения различных структур данных}

\subsection{Массив строк}
Ниже продемонстрировано объявление массивов строк, кол-во элементов в данных массивах равно количеству пользователей, длина каждой строки равна длинне слова:
\begin{lstlisting}[language=Fortran]
integer, parameter              :: USER_AMOUNT = 14, 
	LN_LEN = 15, FN_LEN = 10, MN_LEN = 15

character(:), allocatable     :: input_file, output_file, format
integer                       :: In, Out, IO, i

character(LN_LEN, kind=CH_)   :: Last_Names(USER_AMOUNT) = ""
character(FN_LEN, kind=CH_)   :: First_Names(USER_AMOUNT) = ""
character(MN_LEN, kind=CH_)   :: Middle_Names(USER_AMOUNT) = ""
\end{lstlisting}

Для обработанных данных были объявлены аналогичные массивы выделяемой размерности, так как до окончания обработки неизвестно кол-во строк в результирующих массивах:
\begin{lstlisting}[language=Fortran]
character(LN_LEN, kind=CH_), allocatable :: Last_Names_Res(:)
character(FN_LEN, kind=CH_), allocatable :: First_Names_Res(:)
character(MN_LEN, kind=CH_), allocatable :: Middle_Names_Res(:)
\end{lstlisting}

Обработка массива, создания маски подходящих элементов, которых нужно упаковать в результирующие массивы, упаковка результирующих массивов:
\begin{lstlisting}[language=Fortran]
do concurrent (i=USER_AMOUNT:2:-1)
Users_Mask(i) = .not.(Any(First_Names(:i-1) == First_Names(i)))
end do

First_Names_Res = Pack(First_Names, Users_Mask)
Last_Names_Res = Pack(Last_Names, Users_Mask)
Middle_Names_Res = Pack(Middle_Names, Users_Mask)
\end{lstlisting}

При использовании процедуры Pack реализуется нерегулярный доступ к памяти, используется по причине необходимости работать с исходным массивом согласно условиям задачи. При выборке всех имен пользователей от первого и до необходимо - регулярный, так как данные сплошные.

\subsection{Массив символов}
Ниже продемонтрировано объявление двумерных массивов, количество символов в первом индексе для достижения регулярного доступа к памяти во время сравнения символов при поиске повторений массива имен пользователей:
\begin{lstlisting}[language=Fortran]
integer, parameter        :: USER_AMOUNT = 14, LN_LEN = 15, 
	FN_LEN = 10, MN_LEN = 15

character(:), allocatable :: input_file, output_file
integer                   :: In, Out

character(kind=CH_)    :: Last_Names(LN_LEN, USER_AMOUNT) = ""
character(kind=CH_)    :: First_Names(FN_LEN, USER_AMOUNT) = ""
character(kind=CH_)    :: Middle_Names(MN_LEN, USER_AMOUNT) = ""
\end{lstlisting}

Для обработанных данных были объявлены аналогичные массивы выделяемой размерности, так как до окончания обработки неизвестно кол-во строк в результирующих массивах:
\begin{lstlisting}[language=Fortran]
character(kind=CH_), allocatable      :: Last_Names_Res(:, :)
character(kind=CH_), allocatable      :: First_Names_Res(:, :)
character(kind=CH_), allocatable      :: Middle_Names_Res(:, :)
\end{lstlisting}

Подпрограмма удаления элементов по условию, цикл нужен для формирования массива индексов подходящих элементов, по которым затем они переносятся в результирующие массивы:
\begin{lstlisting}[language=Fortran]
subroutine Handle_User_List(Last_Names, First_Names, Middle_Names, 
	Last_Names_Res, First_Names_Res, Middle_Names_Res)
character(kind=CH_)                 :: Last_Names(:, :), 
	First_Names(:, :), 
Middle_Names(:, :)
character(kind=CH_), allocatable    :: Last_Names_Res(:, :), 
	First_Names_Res(:, :), Middle_Names_Res(:, :)
logical                             :: Users_Mask(USER_AMOUNT)
integer                             :: result_len, i, j
integer, parameter                  :: INDEXES(*) = [(i, i=1, 
	USER_AMOUNT)]
integer, allocatable                :: Unique_Pos(:)

intent (in)       Last_Names, First_Names, Middle_Names
intent (out)      Last_Names_Res, First_Names_Res, Middle_Names_Res

do concurrent (i=User_Amount:2:-1)
Users_Mask(i) = .not.(Any([(All(First_Names(:, j) == 
	First_Names(:, i)), j=1, i-1)]))
end do

result_len = Count(Users_Mask)
Unique_Pos = Pack(INDEXES, Users_Mask)
allocate (Last_Names_Res(LN_LEN, result_len), 
	First_Names_Res(FN_LEN, result_len), 
	Middle_Names_Res(MN_LEN, result_len))

do concurrent (i=1:result_len)
	First_Names_Res(:, i) = First_Names(:, Unique_Pos(i))
	Last_Names_Res(:, i) = Last_Names(:, Unique_Pos(i))
	Middle_Names_Res(:, i) = Middle_Names(:, Unique_Pos(i))
end do
\end{lstlisting}

Данные хранятся в массиве по строкам, следовательно регулярный доступ будет осуществлен при работе с ними. Как и с массивом строк, реализовать векторизацию при заполнении невозможно.

\subsection{Структура массивов}
Объявление структуры массивов:
\begin{lstlisting}[language=Fortran]
integer, parameter :: USER_AMOUNT   = 14
integer, parameter :: LN_LEN        = 15
integer, parameter :: FN_LEN        = 10
integer, parameter :: MN_LEN        = 15

type Users
   character(LN_LEN, kind=CH_), allocatable   :: Last_Names(:)
   character(FN_LEN, kind=CH_), allocatable   :: First_Names(:)
   character(MN_LEN, kind=CH_), allocatable   :: Middle_Names(:)
end type Users
\end{lstlisting}

Была выбрана структура массивов, так как искать нужно исходя из задания только по одному полю, выгодно значения данного поля расположить сплошными данными в одном массиве для регулярного доступа к памяти. Реализовано формирование двоичного файла записей из входного файла:
\begin{lstlisting}[language=Fortran]
subroutine Create_Data_File(Input_File, Data_File)
	character(*), intent(in)   :: Input_File, data_file
	
	integer                    :: In, Out, IO, i
	character(:), allocatable  :: format
	
	character(LN_LEN, kind=CH_) :: Last_Names(USER_AMOUNT)
	character(FN_LEN, kind=CH_) :: First_Names(USER_AMOUNT)
	character(MN_LEN, kind=CH_) :: Middle_Names(USER_AMOUNT)
	
	open (file=Input_File, encoding=E_, newunit=In)
		format = '(3(a, 1x))'
		read (In, format, iostat=IO) (Last_Names(i), 
			First_Names(i), Middle_Names(i), i=1, 
			USER_AMOUNT)
		call Handle_IO_status(IO, 
			"init arrays with formatted file")
	close (In)
	
	open (file=Data_File, form="unformatted", 
		newunit=Out, access="stream")
		write (Out, iostat=IO) Last_Names, First_Names, 
			Middle_Names
		call Handle_IO_status(IO, 
			"creating unformatted file by index")
	close (Out)
end subroutine Create_Data_File
\end{lstlisting}

И функция чтения данных из двоичного файла в структуры массивов:
\begin{lstlisting}[language=Fortran]
function Read_Users_List(Data_File) result(Usrs)
	type(Users)                :: Usrs
	character(*), intent(in)   :: Data_File
	integer                    :: In, IO
	
	allocate (Usrs%Last_Names(USER_AMOUNT), 
		Usrs%First_Names(USER_AMOUNT), 
		Usrs%Middle_Names(USER_AMOUNT))
	open (file=Data_File, form="unformatted", 
		newunit=In, access="stream")
		read (In, iostat=IO) Usrs%Last_Names, 
			Usrs%First_Names, Usrs%Middle_Names
		call Handle_IO_status(IO, "init objects")
	close (In)
end function Read_Users_List
\end{lstlisting}

Реализовано нахождение нужных элементов с помощью рекурсивный процедуры и их упаковывание:
\begin{lstlisting}[language=Fortran]
pure function Handle_Users_List(Usrs) result(Usrs_Res)
type(Users)              :: Usrs
type(Users)              :: Usrs_Res
logical                  :: Users_Mask(USER_AMOUNT)
integer                  :: i

intent (in)              Usrs

do concurrent (i=USER_AMOUNT:2:-1)
	Users_Mask(i) = .not.(Any(Usrs%First_Names(:i-1) == 
		Usrs%First_Names(i)))
end do

Usrs_Res%First_Names = Pack(Usrs%First_Names, Users_Mask)
Usrs_Res%Last_Names = Pack(Usrs%Last_Names, Users_Mask)
Usrs_Res%Middle_Names = Pack(Usrs%Middle_Names, Users_Mask)
end function Handle_Users_List
\end{lstlisting}

\subsection{Динамический список}

Объявляем структуру узла списка:
\begin{lstlisting}[language=Fortran]
integer, parameter :: LN_LEN   = 15
integer, parameter :: FN_LEN   = 10
integer, parameter :: MN_LEN   = 15

type User
	character(LN_LEN, kind=CH_)    :: Last_Name    = ""
	character(FN_LEN, kind=CH_)    :: First_Name   = ""
	character(MN_LEN, kind=CH_)    :: Middle_Name  = ""
	type(User), pointer            :: next         => Null()
end type User
\end{lstlisting}

Реализуем обход списка и удаление неподходящих узлов с помощью рекурсивных процедур:
\begin{lstlisting}[language=Fortran]
recursive subroutine Handle_Users_List(Users)
	type(User), pointer              :: Users
	
	if (Associated(Users)) then
	call Delete_Dublicates(Users, Users%next)
	call Handle_Users_List(Users%next)
	end if
end subroutine Handle_Users_List

recursive subroutine Delete_Dublicates(current, current_next)
	type(User), pointer            :: current_next, current, tmp
	
	if (Associated(current_next)) then
		if (current_next%First_Name == 
			current%First_Name) then
			tmp => current_next
			current_next => current_next%next
			deallocate(tmp)
			call Delete_Dublicates(current, current%next)
	else
		call Delete_Dublicates(current, current_next%next)
		end if
	end if
end subroutine Delete_Dublicates
\end{lstlisting}

При использовании динамического списка неизвестной длины применять о векторизацию и регулярный доступ к памяти невозможно, каждый элемент списка находится в случайном месте в памяти.
\newpage
\section{Сравнение реализаций}

В таблице 1 приведен сравнительный анализ различных реализаций лабораторной работы:
\begin{table}[h]
	\centering
	\begin{tabular}{|p{18ex}|p{11ex}|p{12ex}|p{13ex}|p{18ex}|}
		\hline
		\centering{-}&\bfseries{Массив строк}&\bfseries{Массив символов}&\bfseries{Структура массивов}&\bfseries{Динамический список}   \\
		\hline
		\bfseries{Cплошные данные} & Да & Да & Да & Нет \\
		\hline
		\bfseries{Регулярный доступ} & При обработке & При обработке & При обработке & Нет \\
		\hline
		\bfseries{Векторизация} & Нет & Нет & Нет & Нет \\
		\hline
		\bfseries{Потенциальная векторизация} & При обработке & При обработке & При обработке & Нет \\
		\hline
	\end{tabular}
	\caption{Сравнение реализаций}
\end{table}

Исходя из анализа для выполнения подходят все варианты, кроме динамических списков, как самых медленных для обработки, и неэффективных, если мы заранее знаем размер списка. Среди остальных вариантов наиболее предпочтительна структура массивов, позволяющие сгруппировать данные в одну строкуктуру, повышая уровень абстракции. Также реализация массива строк более предпочтительна перед массивом символов, так как второе решение повышает сложность кода и требует большой компетенции от разработчика при написании и поддержке кода под современных архитектуры.

\newpage
\begin{center}\specialsection{Вывод}\end{center}

Поставленная задача была реализована несколькими способами, среди которых бели выбраны алгоритмы, наиболее подходищие для выполнения программного кода на современных архитектурах. Наиболее подходящим решением оказалась реализация структуры массивов. Поставленная цель выполнена.
\end{document}